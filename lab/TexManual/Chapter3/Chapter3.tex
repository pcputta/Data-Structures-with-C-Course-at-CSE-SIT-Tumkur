\chapter{Infix to Postfix Conversion}
\section*{Question}
{\fontfamily{ppl}\selectfont \textbf{Write a C program to convert the given infix expression to postfix expression.
}}

\section*{C Code}

\begin{Verbatim}
#include <stdio.h>
#include <stdlib.h>
#include <string.h>

#define STK_SIZE 10

void fnPush(char [], int*, char);
char fnPop(char [], int*);
int fnPrecd(char);

int main()
{
	int i, j=0;
	char acExpr[50], acStack[50], acPost[50], cSymb;
	int top = -1;
	
	printf("\nEnter a valid infix expression\n");
	scanf("%s", acExpr);
	
	fnPush(acStack, &top, '#');
	for(i=0;acExpr[i]!='\0'; ++i)
	{
		cSymb = acExpr[i];
		if(isdigit(cSymb))
		{
			fnPush(acStack, &top, cSymb);
		}
		else if(cSymb == '(')
		{
			fnPush(acStack, &top, cSymb);	
		}
		else if(cSymb == ')')
		{
			while(acStack[top] != '(')
			{
				acPost[j++] = fnPop(acStack, &top);
			}
			fnPop(acStack, &top);
		}
		else
		{
			while(fnPrecd(acStack[top]) >= fnPrecd(cSymb))
			{
				acPost[j++] = fnPop(acStack, &top);
			}
			fnPush(acStack, &top, cSymb);
		}

	}
	while(acStack[top] != '#')
	{
		acPost[j++] = fnPop(acStack, &top);
	}
	acPost[j] = '\0';
	
	printf("\nInfix Expression is %s\n", acExpr);
	printf("\nPostfix Expression is %s\n", acPost);
	return 0;	
}

void fnPush(char Stack[], int *t , char elem)
{
	*t = *t + 1;
	Stack[*t] = elem;
	
}

char fnPop(char Stack[], int *t)
{
	char elem;
	elem = Stack[*t];
	*t = *t -1;
	return elem;
}

int fnPrecd(char ch)
{
	switch(ch)
	{
		case '#' : 	return -1;	
		case '(' : 	return 0;
		case '+' : 	
		case '-' : 	return 1;
		case '*' : 	
		case '/' : 	return 2;
	}
}
\end{Verbatim}

\section*{Output}
\begin{Verbatim}



\end{Verbatim}

