\chapter{Singly Linked List Applications}
\section{Stack using SLL}
\textbf{=================================}

{\fontfamily{qbk}\selectfont { \textit{\textbf{Write a C program to implement a STACK using singly linked list.}}} \\
}

\section*{C Code}

\lstinputlisting[language={[ANSI]C}, caption=09aStackLL.c]{Programs/09aStackLL.c}

\section*{Output}
\textbf{=================================}
\lstinputlisting[language={[ANSI]C}, caption=out9a.c]{Programs/out9a.c}
\pagebreak
\section{Queue using SLL}
\textbf{=================================}

{\fontfamily{qbk}\selectfont { \textit{\textbf{Write a C program to implement a QUEUE using singly linked list.}}} \\
}

\section*{C Code}

\lstinputlisting[language={[ANSI]C}, caption=09bQueueLL.c]{Programs/09bQueueLL.c}

\section*{Output}
\textbf{=================================}
\lstinputlisting[language={[ANSI]C}, caption=out9b.c]{Programs/out9b.c}

\pagebreak
\section{Polynomial Addition}
\textbf{=================================}

{\fontfamily{qbk}\selectfont {\textit{\textbf{Write a C program to implement addition of two polynomials using singly linked list..}}} \\
}

\section*{C Code}

\lstinputlisting[language={[ANSI]C}, caption=09cPolynomial.c]{Programs/09cPolynomial.c}

\section*{Output}
\textbf{=================================}
\lstinputlisting[language={[ANSI]C}, caption=out9c.c]{Programs/out9c.c}